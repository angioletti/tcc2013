\documentclass[a4paper]{book}
\usepackage{graphicx}
\usepackage{caption}
\usepackage{subcaption}
\usepackage[brazil]{babel}
\usepackage[utf8]{inputenc}
\usepackage{fullpage}
%
%
% DICAS GERAIS:
%	* LEIA O TEXTO ESCRITO
%	* ATIVE OS REVISORES GRAMATICAIS
%	* PREOCUPE-SE INICIALMENTE COM O CONTEÚDO, DEIXE PARA DEPOIS A ESTÉTICA E A EDIÇÃO
%	* SEJA OBJETIVO E EVITE REDUNDÂNCIA
%	* NUMERE AS EQUAÇÕES
%	* PARA AS LISTAS, TENTE USAR AS FUNÇÕES AUTOMÁTICAS DOS EDITORES
%	* IDEM PARA AS REFERÊNCIAS
%
%
\begin{document}

\chapter{Introdução}
% Qual o problema a ser resolvido pelo trabalho?
% Por que ele é importante (contexto)?
% Quais as contribuições do trabalho para a solução do Problema?

	\section{Enquadramento do Trabalho}
	Inserida no campo de processamento digital, e consideravelmente relevante no campo de visão computacional (citarfonte), a área de qualidade de imagem é um campo de pesquisa que se preocupa em encontrar formas automáticas de avaliar a qualidade de uma imagem em vários aspectos.

	É interessante que se considere a avaliação qualitativa de uma imagem dentro de seu contexto de aplicação, assim, uma imagem considerada de boa qualidade para um determinado uso não necessariamente o é para outro uso qualquer. Tomemos o exemplo simples da identificação de um sujeito em uma imagem. Nesse contexto, a seção de interesse da imagem é o rosto do sujeito em questão, não interessando para nós o fundo, ou a roupa desse indivíduo. Para essa aplicação, o fundo poderia estar fora de foco, ou a imagem ter sido produzida reproduzindo apenas o indivíduo do pescoço para cima.

	Essa mesma imagem, com fundo fora de foco e enquadramento restrito não é uma boa imagem para a visão computacional de uma máquina que procura identificar um ser humano. Para tanto, essa máquina precisa extrair da imagem algumas características que identificam um ser humano: é bípede? tem membros superiores? E o contexto é relevante para diferenciá-lo de um cartaz, por exemplo.

	Em se tratando da aferição da qualidade de uma imagem, nenhuma máquina, até hoje, consegue imitar satisfatoriamente o olho humano, ou a completude do sistema visual humano (HVS, do inglês {\em Humam Visual System}) que, determinando áreas de interesse no campo visual, ignora defeitos consideráveis em áreas de menor interesse e é extremamente sensível a defeitos mínimos na área de interesse de um mesmo campo visual. Sendo assim, o melhor aferidor da qualidade de uma imagem para consumo humano é o homem.

	Em nossa atualidade, quando sistemas digitais de segurança geram milhares de horas de vídeo por dia, todos nós consumimos videos e imagens digitais cotidianamente, ter uma banca de pessoas para avaliar a qualidade das imagens produzidas a cada minuto torna-se impraticável, e se praticável fosse, seria extremamente tedioso. Na intenção de melhorar serviços e produtos, o campo de pesquisa em qualidade de imagem se preocupa em gerar modelos matemáticos que possa ser utilizados automaticamente para determinar a qualidade de uma imagem qualquer, dado o contexto de seu uso provável: consumo humano (edição de imagens, fotos de família, impressão dessas fotos, etc), ou consumo digital: identificação automática de faces, aferição de qualidade de produto em linha de produção, mapeamento de ambiente e tantas outras aplicações da visão computacional.

	Na busca pelo melhor modelo de aferição de qualidade, os pesquisadores não têm outra alternativa que não testar esses modelos comparativamente a notas atribuídas por pessoas. Assim, inúmeros testes foram elaborados com a intenção de capturar a percepção humana sobre a qualidade de uma determinada imagem. Um desses testes, que será muito comentado neste trabalho é a Pontuação Média de Opinião (do jargão inglês {\em Mean Opinion Score ou MOS}). Nesse teste, em geral, dá-se a um ser humano não treinado em aferição de qualidade de imagem um conjunto de imagens para teste e uma folha com uma escala pautada. O sujeito é então solicitado a dar uma nota para cada uma das imagens do conjunto, riscando na escala a sua avaliação (inserir figura).

	Via de regra essa escala é divida em cinco pontos equidistantes (do menor para o maior): muito ruim ($0,0$, zero), ruim ($2,5$), razoável ($5,0$), bom ($7,5$) e excelente ($10,0$). Quaisquer variações de avaliação tornam-se números dentro desses intervalos. Contudo, uma imagem pode ser considerada boa, sistematicamente, com nota $6,0$, ou excelente com nota $7,6$. Esses valores estão, obviamente, numericamente muito mais próximos de seus vizinhos inferiores do que da média do valor para aquele intervalo. Assim, a imagem em questão acaba sendo considerada como boa, ainda que esteja muito próxima de ruim.

	\section{Objetivos do Trabalho}
	A proposta desse trabalho é avaliar e comparar a utilização de estatística por classes nessa questão. Ao invés de atribuir-se uma nota numérica a uma imagem que esteja sendo avaliada, classifica-se simplesmente essa imagem em uma categoria (ruim, por exemplo) e utilizando-se de ferramentas estatísticas de operação em classes, para aferição de características estatísticas dessa imagem.

	A comparação das duas abordagens, a tradicional em espaço contínuo e a por estatística de classes, se dá em alguns aspectos: resultado esperado (a imagem é classificada corretamente em ambas as ferramentas?), facilidade de implementação, tempo de avaliação, ...


	\section{Resultados / Contribuições Relevantes}

	% Os resultados/contribuições devem ser enumerados e descritos de forma sucinta (1 parágrafo por resultado, por exemplo). O TCC deve ser organizado em 
	% torno destes resultados, portanto, devem ser muito bem identificados no início.

	\section{Estrutura deste Documento}
	% Guia de leitura do TCC. Apresenta a lógica do TCC e indica o que vai ser encontrado em cada capítulo. É importante que o leitor saiba o que vai
	% encontrar no TCC.
	%


\chapter{Background}

% Aqui é descrita toda a teoria necessária para o entendimento do trabalho. É um capítulo de caráter informativo e não deve ser grande em demasia.
% É um capítulo generalista, onde se fala sobre os principais conceitos da área
% Sub-dividido em:
%  Introdução
%  Desenvolvimento
%  Conclusão
%

\chapter{Estado da Arte}

% Este é um capítudo de revisão de literatura
% Sub-dividido em:
%  Introdução: apresentada a lógica do capítulo e o que será encontrado em cada uma das seções.
%  Desenvolvimento:
%	Qual o problema analisado no trabalho?
%	Por que ele é importante?
%	O que se tem feito no mundo para resolvê-lo?
%	O ideal é se ter um estudo qeu disserte/comente os trabalhos citados, mostrando a sua relação com o problema, as vantagens e desvantagens de cada trabalho
%		citado.
%	Qual a contribuição desse trabalho? É o diferencial da proposta, principalmente, em relação às referências citadas na revisão bibliográfica.
%  Conclusão: faça referência aos pontos importantes do capítulo, que podem ser transversais às seções. Estes pontos serão, muito provavelmente, úteis ao leitor
% 		dos capítulos seguintes. Tente sempre conectar os capítulos textualmente, com 'ganchos' para o próximo assunto.
%
% Este capítulo deve omitir a teoria básica, apresentada no capítulo anterior.
% Sendo uma revisão de literatura, deve apresentar soluções dos diversos autores para o problema apresentado (ou seja, diante do problema exposto, o que tem sido feito 
% para resolvê-lo? Como? Por quem?
% Destaque como a literatura resolve o problema, vantagens e desvantagens.
%
% Este capítulo deve ser finalizado com a indicação da contribuição do trabalho e indicar o diferencial em relação as abordadas no capítulo. Maiores detalhes sobre a
% proposta do trabalho serão apresentados no próximo capítulo.

\chapter{Descrição do Trabalho}

% Esse capítulo aborda a contribuição do trabalho. Se houver necessidade, pode-se incluir uma seção para se destacar 'uma teoria nova' (ou não citada no capítulo anterior)
% e que seja importante para o entendimento da contribuição.
%
% Sub-dividido em:
%  Introdução
%  Desenvolvimento
%  Conclusão 

\chapter{Descrição do Cenário ou ambiente usado para a obtenção dos resultados. E resultados.}

% Aborda todos os procedimentos utilizados para a obtenção dos resultados.
% Se o trabalho for uma simulação, apresentar o cenário de simulação utilizado.
% Se for um trabalho prático, deve-se descrever o setup laboratorial juntamente com as arquiteturas eletrônicas usadas.
% A metodologia utilizada para validar os resultados/contribuições. Experiências realizadas devem ser descritas aqui.
%   Quais foram os experimentos realizados para se chegar nos resultados?
%   Como esses resultados foram validados?
% Os resultados apresentados devem ser apresentados e analisados! Analise resultados, não descreva gráficos.
% Importante é subsidiar o leitor com todas as informações usadas para a obtenção dos resultados (parâmetros devem ser especificados, quando possível, anexar as listagens
% dos resultados.
%
% Sub-dividido em:
%  Introdução
%  Desenvolvimento
%  Conclusão 
%  

\chapter{Conclusões}

% Revisão do trabalho desenvolvido
%   Objetivos do trabalho, conclusões relevantes
% Resultados / contribuições relevantes
% Resultado 1
%   Caracterização do resultado. Justificativa
%   Aspectos positivos e negativos.
% Resultado 2
% Resultado 3
% Fundamental na conclusão: TRABALHOS FUTUROS, identificar novos trabalhos que possam ser desenvolvidos sobre os resultados encontrados; fornecendo, sempre que possível, 
% 	informações e subsídios de como se pode progredir. Tente evitar esboçar uma lista de possibilidades de trabalhos futuros sem indicar como esses podem ser viabilizados
% 	a partir dos resultados encontrados.
% 
% Referências Bibliográficas: em se tratando de Ref. Bib., todas devem estar citadas no texto e listadas de acordo com a ordem de citação! Comumente adota-se o padrão de citação
% de artigos como os do IEEE

\end{document}

