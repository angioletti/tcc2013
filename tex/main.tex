\documentclass[a4paper]{book}
\usepackage{graphicx}
\usepackage{caption}
\usepackage{subcaption}
\usepackage[brazil]{babel}
\usepackage[utf8]{inputenc}
\usepackage{fullpage}
%
%
% DICAS GERAIS:
%	* LEIA O TEXTO ESCRITO
%	* ATIVE OS REVISORES GRAMATICAIS
%	* PREOCUPE-SE INICIALMENTE COM O CONTEÚDO, DEIXE PARA DEPOIS A ESTÉTICA E A EDIÇÃO
%	* SEJA OBJETIVO E EVITE REDUNDÂNCIA
%	* NUMERE AS EQUAÇÕES
%	* PARA AS LISTAS, TENTE USAR AS FUNÇÕES AUTOMÁTICAS DOS EDITORES
%	* IDEM PARA AS REFERÊNCIAS
%
%
\begin{document}

\chapter{Introdução} \label{chap:intro}

%% Histórico

	Ao longo dos últimos quarenta anos, com a difusão de tecnologias eletrônicas modernas, a palavra ``digital'' tem se tornado cada vez mais lugar comum. Ao longo desse período, a tecnologia baseada em máquinas de cálculo automatizado que ocupavam prédios inteiros e eram operadas por poucos (para fins restritos) foi permeando nossa cultura global e ganhando destaque em vários campos do conhecimento e da vida humanas. Com o passar do tempo, as suas aplicações foram ganhando horizontes, os computadores passaram a ser pessoais e cada vez mais amplamente utilizados, contudo, ainda tinham aplicações específicas em ambientes de pesquisa e para atividades profissionais, e eram restritos a mesas e ambientes reservados. Em paralelo, a indústria de jogos eletrônicos começa a florescer, se fazendo presente nas residências através de seus consoles. Todos aproveitavam a evolução tecnológica apresentada pelo advento dos transistores e microprocessadores.

	Adiante mais alguns anos, e a internet aparece, os microprocessadores estão mais ``micro'' e mais ``processadores'', os consoles de video-game mais elaborados e seus clientes cada vez mais exigentes. Os microprocessadores vão transitando de máquinas estáticas para aplicações móveis: celulares e câmeras digitais (para foto e vídeo) tornam-se quase onipresentes. Mídias digitais tornam-se tão importantes quanto protocolos de comunicação, redes sociais movimentam a opinião pública e servem de plataforma a revoluções ~\cite{revoltaEgito}. Muitos cidadãos globais têm contas em várias redes sociais e máquinas portáteis em seus bolsos, prontas a fazer um vídeo ou uma foto e postá-los para apreciação popular. Ao que se faz possível, com alguns poucos cliques e nenhum custo com passagem ou hospedagem, termos acesso à realidade atualizada de diversas partes do globo, em todos os idiomas possíveis, a qualquer hora.

	O volume de dados gerado é tão grande que plataformas como o Flickr (da americana Yahoo) abrigava, em 2011, mais de seis bilhões de imagens digitais ~\cite{flickr6bi}, e o crescimento estimado para o crescimento de seu banco de imagens nos anos seguintes foi de um bilhão por ano --- e essa é apenas uma de várias plataformas de hospedagem de publicação de mídias ~\cite{500px} \cite{instagram}. Outro exemplo, para publicação de vídeos, é o Youtube, onde seus mais de um milhão de usuários únicos assitem mais de seis bilhões de horas de vídeo por mês. Outras plataformas de publicação de vídeo também têm números expressivos, como o caso do Vimeo, com mais de 200 \emph{petabytes} de vídeos executados em 2012 ~\cite{vimeo}. Além de serviços de hospedagem de conteúdo disponíveis publicamente, mais recentemente surgiram serviços de \emph{streamming} de filmes e séries, como o Netflix, já com mais de 40 milhões de usuários em 41 países, disponibilizando mais de um bilhão de horas de vídeo~\cite{netflix}.

	Esses números representam um desafio para a indústria, que se responsabiliza por receber, armazenar e distribuir esses dados sob demanda, para todo o globo.

	Voltemos um pouco à história de alguns parágrafos acima. Outra entidade que acompanha esse desenvolvimento, e que o antecede em algumas décadas é a televisão, cujo pai aclamado é Philo Farnsworth, entre tantos outros engenheiros e contribuições~\cite{tvhist}. Desde 1929 existe programação regular para TV sendo projetada no espaço, proveniente de ambos os lados do Atlântico (os Estados Unidos e o Reino Unido começaram a produzir programação regularmente na segunda metade de 1929)~\cite{tvhist}. 

	Com o nascimento dessa tecnologia, começou-se a discutir a necessidade de avaliar a qualidade da imagem recebida em aparelhos de TV, e artigos como ``\emph{ Television images: an analysis of their essential qualities}'', publicado por Jesty e Wintch, em 1937\cite{Jesty1937}, aparecem. Winch, começa seu artigo de 1953~\cite{Winch1953} com a afirmação (em tradução livre) ``a adição de cores à televisão traz muitos novos problemas a um assunto já complexo". E já em 1940, Peter Goldmark e John Dyer ~\cite{Goldmark1940} apresentam quais características são mais importantes ao determinar-se a qualidade de uma imagem (para TV): definição, faixa de contraste, ângulo de visualização (\textbf{gradation?}), brilho, efeitos da frequência de varredura (\emph{flickering}) , distorção geométrica, tamanho, cor e ruído. Algumas dessas características viriam a se tornar objetos de estudo da AQI nos anos vindouros, bem como algumas delas seriam bases para o cálculo de métricas de qualidade de imagem (fotografia e vídeo) --- falaremos de algumas delas nesse trabalho.

	Dado que a parte visual de um vídeo é constituída de imagens paradas em sequência, a avaliação da qualidade de vídeo e de imagem andam entrelaçadas desde o início. Tanto o é, que a International Telecommunications Union --- ITU (União Internacional de Telecomunicações, tradução livre) não faz distinção em seus documentos de padronização de qualidade entre vídeo e imagem~\cite{itut2004}; e seu grupo especializado para esse fim é chamado \emph{Video Quality Experts Group} (VQEG, Grupo de Peritos em Qualidade de Vídeo, tradução livre). Em ~\cite{itut2004}, encontramos recomendações para avaliação de qualidade de vídeo que podem ser também apliadas a imagens. Nesse trabalho seguiremos essas recomendações.

	Vê-se que, com tanta demanda por imagem e vídeo, e tráfego destes, é necessário que se encontre uma forma de armazená-los eficientemente, acessá-los confiávelmente e garantir que o usuário final terá a qualidade esperada, ainda que compressões e descarte de informações sejam necessários. Enquanto alguém preocupado com a compressão de uma imagem se perguntaria ``qual a menor quantidade de informação necessária para que se mantenha a completude da mensagem (no nosso caso, imagem)?'', um pesquisador atual de AQI se pergunta: ``como aferir a qualidade de uma imagem digital? Dado que a necessidade de reduzir a quantidade de informação é premente, o que garante qualidade percebida?''

	Essa é uma pergunta razoávelmente complexa, já que envolve conceitos abstratos e subjetivos. Uma boa imagem para uma aplicação não o é, necessáriamente para outra. Um exemplo simples é a aquisição de vídeos de segurança em comparação com a aquisição de vídeo para entretenimento. No primeiro caso, a qualidade mínima e suficiente é aquela que garante a identificação de um possível infrator; na segunda as exigências são mais altas (ninguém sentiria prazer em assistir toda a saga de Star Wars em preto e branco e com todo o ruído e baixa resolução que são aceitáveis para câmeras de segurança). É claro, não se pode deixar de lado as considerações sobre custo-benefício: no primeiro caso a resolução tem que ser mínima e suficiente para a identificação de um eventual infrator, mas também tem que ocupar pouco espaço em disco, já que câmeras de segurança, em sua maioria, funcionam continuamente. Quanto a Star Wars, o tamanho é fixo, uma vez terminada a edição; e a garantia da qualidade do produto final significa aumento de lucros em bilheterias pelo mundo afora.

	Tradicionalmente, existem duas formas de se construir algoritmos de AQI que possam determinar automaticamente a adequação de uma imagem para consumo humano~\cite{Chandler2013}: algoritmos baseados no sistema visual humano (SVH) e algoritmos baseados em sessões de avaliação.

	Algoritmos baseados em SVH levam em consideração a física de todo um sistema biológico, bem como a parte psicológica da percepção visual. Esse trabalho não tratará desse tipo de algoritmo e o leitor é direcionado aos trabalhos de ~\cite{Takemura2002} e ~\cite{Winkler-2005-Wiley}\ para maiores informações. 
	
	A segunda abordagem, que será alvo de estudo nesse trabalho, não aborda a psicofísica e se concentra em características da imagem que sejam relevantes. Para obter esses dados, sessões de avaliação são organizadas, onde pessoas são questionadas a respeito da qualidade de um conjunto de imagens. Após coletados os dados, o pesquisador tenta produzir um modelo que tenha como saída uma nota aproximada daquela dada pelos entrevistados. Obviamente, quanto menor o erro entre o sistema modelado e a avalição subjetiva dos indivíduos entrevistados, melhor o modelo.

	Aqui trabalharemos com duas bases de imagens distintas, proveniente de grupos de pesquisa independentes, que coletaram avaliação para as imagens constantes em suas bases de forma muito similar. Trataremos dessa similaridade, e eventual diferença no cap /ref{}. Ambos os grupos de pesquisa, seguem o que é de praxe na área, e tratam seus procedimentos de avaliação estatisticamente com muito critério, segundo as recomendações do ITU e o usualmente praticado na área.

	O questionamento que propomos é justamente sobre as ferramentas estatísticas utilizadas no tratamento desses dados e na validação das métricas em uso. Entendemos que, dada a natureza dos experimentos de avaliação e a forma como as notas são atribuídas às imagens, ferramentas de estatística em classes seriam mais adequadas para a análise dos resultados, e consequente pondereção de qualidade das métricas desenvolvidas sobre esses dados. Atualmente, as ferramentas estatísticas utilizadas nesse contexto são as mesmas utilizadas para tratar dados intrinsecamente não-categóricos.

	

	Esse documento está estruturado em X partes, resumidamente descritas abaixo:

\begin{description}
\item{\textbf{Introdução:} } breve histórico da área de Avaliação de Qualidade de Imagem, breve descrição do objetivo do trabalho e descrição suscinta das ferramentas utilizadas
\item{\textbf{Antecedentes:} } detalhamos melhor nossa proposta de trabalho, situamos o leitor quanto às abordagens estatísticas utilizadas na área e nossas ferramentas de comparação.
\item{\textbf{Procedimentos Experimentais:} } apresentamos as bases de imagem que serão utilizadas, os métodos utilizados e resultados preliminares de comparação com a literatura existente.
\item{\textbf{Análise Categórica:} } aprofundamos a explicação das ferramentas utilizadas, aplicamos essas ferramentas aos dados apresentados nos capítulos anteriores e efetuamos comparações de sua validade.
\item{\textbf{Conclusão:} } considerações finais sobre o trabalho e possíveis caminhos a serem tomados a partir das conclusões apresentadas.
\item{\textbf{Referências:} } lista de documentos que serviram de base para a produção desta obra.
\end{description}

% Qual o problema a ser resolvido pelo trabalho?
% Por que ele é importante (contexto)?
% Quais as contribuições do trabalho para a solução do Problema?

	\section{Enquadramento do Trabalho}
	Inserida no campo de processamento digital, e consideravelmente relevante no campo de visão computacional (citarfonte), a área de qualidade de imagem é um campo de pesquisa que se preocupa em encontrar formas automáticas de avaliar a qualidade de uma imagem em vários aspectos.

	É interessante que se considere a avaliação qualitativa de uma imagem dentro de seu contexto de aplicação, assim, uma imagem considerada de boa qualidade para um determinado uso não necessariamente o é para outro uso qualquer. Tomemos o exemplo simples da identificação de um sujeito em uma imagem. Nesse contexto, a seção de interesse da imagem é o rosto do sujeito em questão, não interessando para nós o fundo, ou a roupa desse indivíduo. Para essa aplicação, o fundo poderia estar fora de foco, ou a imagem ter sido produzida reproduzindo apenas o indivíduo do pescoço para cima.

	Essa mesma imagem, com fundo fora de foco e enquadramento restrito não é uma boa imagem para a visão computacional de uma máquina que procura identificar um ser humano. Para tanto, essa máquina precisa extrair da imagem algumas características que identificam um ser humano: é bípede? tem membros superiores? E o contexto é relevante para diferenciá-lo de um cartaz, por exemplo.

	Em se tratando da aferição da qualidade de uma imagem, nenhuma máquina, até hoje, consegue imitar satisfatoriamente o olho humano, ou a completude do sistema visual humano (HVS, do inglês {\em Humam Visual System}) que, determinando áreas de interesse no campo visual, ignora defeitos consideráveis em áreas de menor interesse e é extremamente sensível a defeitos mínimos na área de interesse de um mesmo campo visual. Sendo assim, o melhor aferidor da qualidade de uma imagem para consumo humano é o homem.

	Em nossa atualidade, quando sistemas digitais de segurança geram milhares de horas de vídeo por dia, todos nós consumimos videos e imagens digitais cotidianamente, ter uma banca de pessoas para avaliar a qualidade das imagens produzidas a cada minuto torna-se impraticável, e se praticável fosse, seria extremamente tedioso. Na intenção de melhorar serviços e produtos, o campo de pesquisa em qualidade de imagem se preocupa em gerar modelos matemáticos que possa ser utilizados automaticamente para determinar a qualidade de uma imagem qualquer, dado o contexto de seu uso provável: consumo humano (edição de imagens, fotos de família, impressão dessas fotos, etc), ou consumo digital: identificação automática de faces, aferição de qualidade de produto em linha de produção, mapeamento de ambiente e tantas outras aplicações da visão computacional.

	Na busca pelo melhor modelo de aferição de qualidade, os pesquisadores não têm outra alternativa que não testar esses modelos comparativamente a notas atribuídas por pessoas. Assim, inúmeros testes foram elaborados com a intenção de capturar a percepção humana sobre a qualidade de uma determinada imagem. Um desses testes, que será muito comentado neste trabalho é a Pontuação Média de Opinião (do jargão inglês {\em Mean Opinion Score ou MOS}). Nesse teste, em geral, dá-se a um ser humano não treinado em aferição de qualidade de imagem um conjunto de imagens para teste e uma folha com uma escala pautada. O sujeito é então solicitado a dar uma nota para cada uma das imagens do conjunto, riscando na escala a sua avaliação (inserir figura).

	Via de regra essa escala é divida em cinco pontos equidistantes (do menor para o maior): muito ruim ($0,0$, zero), ruim ($2,5$), razoável ($5,0$), bom ($7,5$) e excelente ($10,0$). Quaisquer variações de avaliação tornam-se números dentro desses intervalos. Contudo, uma imagem pode ser considerada boa, sistematicamente, com nota $6,0$, ou excelente com nota $7,6$. Esses valores estão, obviamente, numericamente muito mais próximos de seus vizinhos inferiores do que da média do valor para aquele intervalo. Assim, a imagem em questão acaba sendo considerada como boa, ainda que esteja muito próxima de ruim.

	\section{Objetivos do Trabalho}
	A proposta desse trabalho é avaliar e comparar a utilização de estatística por classes nessa questão. Ao invés de atribuir-se uma nota numérica a uma imagem que esteja sendo avaliada, classifica-se simplesmente essa imagem em uma categoria (ruim, por exemplo) e utilizando-se de ferramentas estatísticas de operação em classes, para aferição de características estatísticas dessa imagem.

	A comparação das duas abordagens, a tradicional em espaço contínuo e a por estatística de classes, se dá em alguns aspectos: resultado esperado (a imagem é classificada corretamente em ambas as ferramentas?), facilidade de implementação, tempo de avaliação, ...


	\section{Resultados / Contribuições Relevantes}

	% Os resultados/contribuições devem ser enumerados e descritos de forma sucinta (1 parágrafo por resultado, por exemplo). O TCC deve ser organizado em 
	% torno destes resultados, portanto, devem ser muito bem identificados no início.

	\section{Estrutura deste Documento}
	% Guia de leitura do TCC. Apresenta a lógica do TCC e indica o que vai ser encontrado em cada capítulo. É importante que o leitor saiba o que vai
	% encontrar no TCC.
	%


\chapter{Background}

% Aqui é descrita toda a teoria necessária para o entendimento do trabalho. É um capítulo de caráter informativo e não deve ser grande em demasia.
% É um capítulo generalista, onde se fala sobre os principais conceitos da área
% Sub-dividido em:
%  Introdução
%  Desenvolvimento
%  Conclusão
%

\chapter{Estado da Arte}

% Este é um capítudo de revisão de literatura
% Sub-dividido em:
%  Introdução: apresentada a lógica do capítulo e o que será encontrado em cada uma das seções.
%  Desenvolvimento:
%	Qual o problema analisado no trabalho?
%	Por que ele é importante?
%	O que se tem feito no mundo para resolvê-lo?
%	O ideal é se ter um estudo qeu disserte/comente os trabalhos citados, mostrando a sua relação com o problema, as vantagens e desvantagens de cada trabalho
%		citado.
%	Qual a contribuição desse trabalho? É o diferencial da proposta, principalmente, em relação às referências citadas na revisão bibliográfica.
%  Conclusão: faça referência aos pontos importantes do capítulo, que podem ser transversais às seções. Estes pontos serão, muito provavelmente, úteis ao leitor
% 		dos capítulos seguintes. Tente sempre conectar os capítulos textualmente, com 'ganchos' para o próximo assunto.
%
% Este capítulo deve omitir a teoria básica, apresentada no capítulo anterior.
% Sendo uma revisão de literatura, deve apresentar soluções dos diversos autores para o problema apresentado (ou seja, diante do problema exposto, o que tem sido feito 
% para resolvê-lo? Como? Por quem?
% Destaque como a literatura resolve o problema, vantagens e desvantagens.
%
% Este capítulo deve ser finalizado com a indicação da contribuição do trabalho e indicar o diferencial em relação as abordadas no capítulo. Maiores detalhes sobre a
% proposta do trabalho serão apresentados no próximo capítulo.

\chapter{Descrição do Trabalho}

% Esse capítulo aborda a contribuição do trabalho. Se houver necessidade, pode-se incluir uma seção para se destacar 'uma teoria nova' (ou não citada no capítulo anterior)
% e que seja importante para o entendimento da contribuição.
%
% Sub-dividido em:
%  Introdução
%  Desenvolvimento
%  Conclusão 

\chapter{Descrição do Cenário ou ambiente usado para a obtenção dos resultados. E resultados.}

% Aborda todos os procedimentos utilizados para a obtenção dos resultados.
% Se o trabalho for uma simulação, apresentar o cenário de simulação utilizado.
% Se for um trabalho prático, deve-se descrever o setup laboratorial juntamente com as arquiteturas eletrônicas usadas.
% A metodologia utilizada para validar os resultados/contribuições. Experiências realizadas devem ser descritas aqui.
%   Quais foram os experimentos realizados para se chegar nos resultados?
%   Como esses resultados foram validados?
% Os resultados apresentados devem ser apresentados e analisados! Analise resultados, não descreva gráficos.
% Importante é subsidiar o leitor com todas as informações usadas para a obtenção dos resultados (parâmetros devem ser especificados, quando possível, anexar as listagens
% dos resultados.
%
% Sub-dividido em:
%  Introdução
%  Desenvolvimento
%  Conclusão 
%  

\chapter{Conclusões}

% Revisão do trabalho desenvolvido
%   Objetivos do trabalho, conclusões relevantes
% Resultados / contribuições relevantes
% Resultado 1
%   Caracterização do resultado. Justificativa
%   Aspectos positivos e negativos.
% Resultado 2
% Resultado 3
% Fundamental na conclusão: TRABALHOS FUTUROS, identificar novos trabalhos que possam ser desenvolvidos sobre os resultados encontrados; fornecendo, sempre que possível, 
% 	informações e subsídios de como se pode progredir. Tente evitar esboçar uma lista de possibilidades de trabalhos futuros sem indicar como esses podem ser viabilizados
% 	a partir dos resultados encontrados.
% 
% Referências Bibliográficas: em se tratando de Ref. Bib., todas devem estar citadas no texto e listadas de acordo com a ordem de citação! Comumente adota-se o padrão de citação
% de artigos como os do IEEE

\end{document}

