\documentclass[a4paper]{book}
\usepackage{graphicx}
\usepackage{caption}
\usepackage{subcaption}
\usepackage[brazil]{babel}
\usepackage[utf8]{inputenc}

%
%
% DICAS GERAIS:
%	* LEIA O TEXTO ESCRITO
%	* ATIVE OS REVISORES GRAMATICAIS
%	* PREOCUPE-SE INICIALMENTE COM O CONTEÚDO, DEIXE PARA DEPOIS A ESTÉTICA E A EDIÇÃO
%	* SEJA OBJETIVO E EVITE REDUNDÂNCIA
%	* NUMERE AS EQUAÇÕES
%	* PARA AS LISTAS, TENTE USAR AS FUNÇÕES AUTOMÁTICAS DOS EDITORES
%	* IDEM PARA AS REFERÊNCIAS
%
%
\begin{document}

\chapter{Introdução}
% Qual o problema a ser resolvido pelo trabalho?
% Por que ele é importante (contexto)?
% Quais as contribuições do trabalho para a solução do Problema?

	\section{Enquadramento do Trabalho}

	%
	%

	\section{Objetivos do Trabalho}

	%
	%

	\section{Resultados / Contribuições Relevantes}

	% Os resultados/contribuições devem ser enumerados e descritos de forma sucinta (1 parágrafo por resultado, por exemplo). O TCC deve ser organizado em 
	% torno destes resultados, portanto, devem ser muito bem identificados no início.

	\section{Estrutura deste Documento}

	% Guia de leitura do TCC. Apresenta a lógica do TCC e indica o que vai ser encontrado em cada capítulo. É importante que o leitor saiba o que vai
	% encontrar no TCC.

\chapter{Background}

% Aqui é descrita toda a teoria necessária para o entendimento do trabalho. É um capítulo de caráter informativo e não deve ser grande em demasia.
% É um capítulo generalista, onde se fala sobre os principais conceitos da área
% Sub-dividido em:
%  Introdução
%  Desenvolvimento
%  Conclusão
%

\chapter{Estado da Arte}

% Este é um capítudo de revisão de literatura
% Sub-dividido em:
%  Introdução: apresentada a lógica do capítulo e o que será encontrado em cada uma das seções.
%  Desenvolvimento:
%	Qual o problema analisado no trabalho?
%	Por que ele é importante?
%	O que se tem feito no mundo para resolvê-lo?
%	O ideal é se ter um estudo qeu disserte/comente os trabalhos citados, mostrando a sua relação com o problema, as vantagens e desvantagens de cada trabalho
%		citado.
%	Qual a contribuição desse trabalho? É o diferencial da proposta, principalmente, em relação às referências citadas na revisão bibliográfica.
%  Conclusão: faça referência aos pontos importantes do capítulo, que podem ser transversais às seções. Estes pontos serão, muito provavelmente, úteis ao leitor
% 		dos capítulos seguintes. Tente sempre conectar os capítulos textualmente, com 'ganchos' para o próximo assunto.
%
% Este capítulo deve omitir a teoria básica, apresentada no capítulo anterior.
% Sendo uma revisão de literatura, deve apresentar soluções dos diversos autores para o problema apresentado (ou seja, diante do problema exposto, o que tem sido feito 
% para resolvê-lo? Como? Por quem?
% Destaque como a literatura resolve o problema, vantagens e desvantagens.
%
% Este capítulo deve ser finalizado com a indicação da contribuição do trabalho e indicar o diferencial em relação as abordadas no capítulo. Maiores detalhes sobre a
% proposta do trabalho serão apresentados no próximo capítulo.

\chapter{Descrição do Trabalho}

% Esse capítulo aborda a contribuição do trabalho. Se houver necessidade, pode-se incluir uma seção para se destacar 'uma teoria nova' (ou não citada no capítulo anterior)
% e que seja importante para o entendimento da contribuição.
%
% Sub-dividido em:
%  Introdução
%  Desenvolvimento
%  Conclusão 

\chapter{Descrição do Cenário ou ambiente usado para a obtenção dos resultados. E resultados.}

% Aborda todos os procedimentos utilizados para a obtenção dos resultados.
% Se o trabalho for uma simulação, apresentar o cenário de simulação utilizado.
% Se for um trabalho prático, deve-se descrever o setup laboratorial juntamente com as arquiteturas eletrônicas usadas.
% A metodologia utilizada para validar os resultados/contribuições. Experiências realizadas devem ser descritas aqui.
%   Quais foram os experimentos realizados para se chegar nos resultados?
%   Como esses resultados foram validados?
% Os resultados apresentados devem ser apresentados e analisados! Analise resultados, não descreva gráficos.
% Importante é subsidiar o leitor com todas as informações usadas para a obtenção dos resultados (parâmetros devem ser especificados, quando possível, anexar as listagens
% dos resultados.
%
% Sub-dividido em:
%  Introdução
%  Desenvolvimento
%  Conclusão 
%  

\chapter{Conclusões}

% Revisão do trabalho desenvolvido
%   Objetivos do trabalho, conclusões relevantes
% Resultados / contribuições relevantes
% Resultado 1
%   Caracterização do resultado. Justificativa
%   Aspectos positivos e negativos.
% Resultado 2
% Resultado 3
% Fundamental na conclusão: TRABALHOS FUTUROS, identificar novos trabalhos que possam ser desenvolvidos sobre os resultados encontrados; fornecendo, sempre que possível, 
% 	informações e subsídios de como se pode progredir. Tente evitar esboçar uma lista de possibilidades de trabalhos futuros sem indicar como esses podem ser viabilizados
% 	a partir dos resultados encontrados.
% 
% Referências Bibliográficas: em se tratando de Ref. Bib., todas devem estar citadas no texto e listadas de acordo com a ordem de citação! Comumente adota-se o padrão de citação
% de artigos como os do IEEE

\end{document}

